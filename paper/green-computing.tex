\documentclass[sigconf]{acmart}

\AtBeginDocument{%
  \providecommand\BibTeX{{%
    Bib\TeX}}}

\setcopyright{acmlicensed}
\copyrightyear{2024}
\acmYear{2024}
\acmDOI{XXXXXXX.XXXXXXX}

\acmConference[ITICSE'24]{ITICSE'24}{June 03--05,  2018}{Milan, IT}
\acmISBN{978-1-4503-XXXX-X/18/06}

\begin{document}

\title{Teaching how to engineer greener software}

% \author{JJ Merelo-Guervós}
% \email{jmerelo@ugr.es}
% \orcid{0000-0002-1385-9741}
% \affiliation{%
%   \institution{Department of Computer Engineering, Automatics and Robotics and CITIC, University of Granada}
%   \city{Granada}
%   \country{Spain}
% }
%
\author{A. U. Thor}
\email{thorau@univ.edu}
\orcid{0000-000}
\affiliation{%
  \institution{Department of XYZ, Edu. U.}
  \city{A City}
  \country{A Country}
}

\renewcommand{\shortauthors}{A. U. Thor}

\begin{abstract}
{\em Green} computing is a general term that describes a host of techniques that
try to minimize the carbon footprint of software applications. As such, it is
not a single body of knowledge, but a series of best practices that help reduce
energy consumption relying on the features of any of the different layers that
are exercised by a software applications. This represents a challenge at the
time of designing a comprehensive syllabus that would help students develop the
series of skills needed to identify energy bottlenecks and eliminate them. In
this poster we will describe the different concepts involved, and how they will
be delivered to guarantee the achievement of learning objectives.
\end{abstract}

\begin{CCSXML}
<ccs2012>
   <concept>
       <concept_id>10011007.10011074.10011081.10011082.10011083</concept_id>
       <concept_desc>Software and its engineering~Agile software development</concept_desc>
       <concept_significance>500</concept_significance>
       </concept>
 </ccs2012>
\end{CCSXML}

\ccsdesc[500]{Software and its engineering~Agile software development}


%%
%% Keywords. The author(s) should pick words that accurately describe
%% the work being presented. Separate the keywords with commas.
\keywords{Software engineering, green computing}

%\received{20 February 2007}
%\received[revised]{12 March 2009}
%\received[accepted]{5 June 2009}

\maketitle



\bibliographystyle{ACM-Reference-Format}
\bibliography{../rpg.bib,../dev.bib, ../agile.bib,../learning-analytics.bib,../pbl.bib,../formative-evaluation.bib}

\end{document}
\endinput

