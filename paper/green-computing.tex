\documentclass[sigconf]{acmart}

\AtBeginDocument{%
  \providecommand\BibTeX{{%
    Bib\TeX}}}

\setcopyright{acmlicensed}
\copyrightyear{2024}
\acmYear{2024}
\acmDOI{XXXXXXX.XXXXXXX}

\acmConference[ITICSE'24]{ITICSE'24}{June 03--05,  2018}{Milan, IT}
\acmISBN{978-1-4503-XXXX-X/18/06}

\begin{document}

\title{Teaching how to engineer greener software}

% \author{JJ Merelo-Guervós}
% \email{jmerelo@ugr.es}
% \orcid{0000-0002-1385-9741}
% \affiliation{%
%   \institution{Department of Computer Engineering, Automatics and Robotics and CITIC, University of Granada}
%   \city{Granada}
%   \country{Spain}
% }
%
\author{A. U. Thor}
\email{thorau@univ.edu}
\orcid{0000-000}
\affiliation{%
  \institution{Department of XYZ, Edu. U.}
  \city{A City}
  \country{A Country}
}

\renewcommand{\shortauthors}{A. U. Thor}

\begin{abstract}
{\em Green} computing is a general term that describes a host of techniques that
try to minimize the carbon footprint of software applications. As such, it is
not a single body of knowledge, but a series of best practices that help reduce
energy consumption relying on the features of any of the different layers that
are exercised by software applications. This represents a challenge at the
time of designing a comprehensive syllabus that would help students develop the
series of skills needed to identify energy bottlenecks and eliminate them. In
this poster we will describe the different concepts involved, and how they will
be delivered to guarantee the achievement of learning objectives.
\end{abstract}

\begin{CCSXML}
<ccs2012>
   <concept>
       <concept_id>10011007.10011074.10011081.10011082.10011083</concept_id>
       <concept_desc>Software and its engineering~Agile software development</concept_desc>
       <concept_significance>500</concept_significance>
       </concept>
 </ccs2012>
\end{CCSXML}

\ccsdesc[500]{Software and its engineering~Agile software development}


%%
%% Keywords. The author(s) should pick words that accurately describe
%% the work being presented. Separate the keywords with commas.
\keywords{Software engineering, green computing, project-based learning}

%\received{20 February 2007}
%\received[revised]{12 March 2009}
%\received[accepted]{5 June 2009}

\maketitle

Green computing \cite{kurp2008green} deals, in general, with reducing the
environmental impact of the creation and use of computing resources. From the
software perspective, it proposes maximizing the amount of work done for every
unit of energy spent. But in order to achieve that, how energy is spent across
all the different computing layers need to be assessed, and understood.

This is why getting the student to achieve a certain amount of understanding of
the different process involved, methodologies needed to carry out that
assessment, and eventually design your code from the ground up or refactoring it
to make it {\em greener} is a challenge.

And it is a challenge that has been recently acknowledged by the joint IEEE/ACM
task force in \cite{cc2020}. Environmental concerns is one of the skills
mentioned in the draft competencies in software engineering as part of the
needed ``behavioral attributes'' as well as in the
Master's degree in Information Systems (IS), as part of the IS Strategy and
Governance competence; the computer engineering set of draft competencies
includes it as part of the Systems Resource Management subject matter. This
again shows the inter-disciplinary content that needs to be considered even if
the focus is on software engineering.

The final objective would be to get the student to understand how choices of
hardware and software platform from the ground up will affect the environmental
impact of the workload that is going to be created or refactored and which best
practices need to be involved to reduce that impact. The syllabus proposed
would, then, would be as follows:\begin{itemize}
\item {\bf Understanding the hardware}: Computing units (CPU, GPU, memory) and their energy profiles. Energy-wise
  heterogeneous architectures. Energy consumption sensors and
  standard APIs to get measurements from them (e.g. RAPL). Interfaces to the
  computing power configuration and administration system (e.g. ACPI). This will
  help the student to understand how the workload exercises different parts of
  the hardware, and why it does so.
\item {\bf Understanding how the workload spends energy}: software profiling,
  methodologies and tools for energy profiling (PowerMeter, pinpoint, hardware meters). This will help the student identify
  bottlenecks from the point of view of performance as well as energy
  consumption; also find out how this consumption scales with workload size.
\item {\bf Refactor for reduction of energy footprint}: once the bottlenecks
  have been identified and specific benchmarks developed to measure the energy
  footprint, the student needs to work across the board to reduce the footprint:
  choosing the computing platform where possible, configuring the application to
  work on specific computing units, choosing the programming language toolchain
  that reduces energy consumption (such as the compiler or interpreter) or
  configuring it, change
  in data structures used to store and process data or leveraging of
  multi-threading or symmetric multiprocessing capabilities.
\end{itemize}

Since most of these items require hands-on experience, we have decided to use
active learning methodologies, organizing the students in groups and
making them develop a project from scratch where they will first measure energy
consumption and then minimize it. The grade can be tied to the
reduction achieved. This way, putting the best practices in green computing
to use, the students will be able to learn the skills and commit them to muscle
memory so that they can be deployed in the future when needed.

\bibliographystyle{ACM-Reference-Format}
\bibliography{../energy.bib,../dev.bib}

\end{document}
\endinput

